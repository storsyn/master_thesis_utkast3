\chapter{Design}

\section{Development of Digital Nudges} 
%Explain how the digital nudges that were tested were designed:
For the development of the digital nudge trial DND method (Mirshc et al. 2018) was partly followed. 

Also got inspiration from earlier studies, conducting similar studies. 

This chapter is devoted to describe the design process of the digital nudges that are investigated in this research. 

For the development of the digital nudges investigated in this study, the DND method by Mircsh et al. was used for inspiration and guiding, meaning that is was partly applied / followed loosely. As this is not an evaluation of the DND method itself, but rather to get some guidance on how to decide for what to test. 

This method consist of three phases: analysing, designing and evaluation. the last phase suggest that one should evaluate the design intervention does not fulfill the desired effects, I will not use perform iterative process, as the evaluation part is what my study focuses at. 

\textbf{Timing and Frequency }
Due to the scope of this master thesis / research, timing and frequency, has not been deeply/widely considered when designing these digital nudges.  That is to say, there was chosen three different times during a day and , during one month the participants were supposed to receive 16 nudges in total. Randomly distributed during the 30 days trial.  

\subsection{Health information}
The health information that was presented through the digital nudges was taken from Helsedirektoratet and FHI, who currently has the leading role of informing (but also motivating?) norwegian residents about physical activity.

\subsection{Randomization and broadcast plan}
16 distinct nudges was randomly distributed over a period of 30 days. As timing was not primary investigation point, randomized, but still within convenient time frames (morning 08-10, noon 12-15, evening 18-20). 

\subsection{Technical implementation}
Sats har selv utviklet et CMS (content management system) (?) for publisering av 



"Selve push meldingen (den som operativsystemet gir bruker, og lever på utsiden av appen); Her bruker vi Google Firebase for å kommunisere med telefonen. Den tar utgangspunkt i Device ID (telefonen) og pusher til Device ID eller IDer (om brukeren er registert med flere devicer). Dette oppfattes som selve push-meldingen. Det er vår serviceplattform som kommuniserer med Google Firebase (og CMSet du kjenner til er det som pusher informasjon inn til vår service plattform). Vår service plattform er forresten ren Microsoft Azure."


