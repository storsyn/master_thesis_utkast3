\chapter{Digital Nudge Intervention Design}
This chapter describes the process of designing, organizing and implementing the digital nudge intervention that was implemented for this study.

\section{Method} 
As stated in the Chapter 2, there has been some contributions into the design of digital nudges. For the development of the digital nudges implemented in the Sats app, we partially followed the proposed DND method of Mirsch et al. The reason we chose it is because it is the currently most defined method to follow for designing digital nudges. However, as it constitutes a method for the overall/complete development of digital nudges and in light of the organizational goals, and we were only supposed to add a digital nudge to an already existing app, it was partly followed… 

\textbf{Phase 1 - Analysing}
R1: Definition of organisational goals - Define organisational goals to be achieved with the digital nudge. 

Organisational goals: want to keep members motivated, aim to help members achieve the the recommendations on physical activity by the authorities/policymakers, 

R2: Definition of desired user behaviour - Define the behaviour the user should perform in light of the organisational goals. 
Target behaviour: regular physical activity 

R3: Analysis of user goals - Analyse the user’s goals. 
Users goals: users have different goals, but they share the intention to be physical active at some level 

R4: Analysis of user characteristics and decision-making process - Analyse the user’s characteristics and impediments to performing the desired behaviour, focussing on heuristics and biases. 
Users characteristics: variation in age, gender, type of motivation, attitude and interest in physical activity,

R5: Definition of technology channel - Analyse the strengths and weaknesses of available technology channels and select the best channel to carry the intervention.
Available technology channels: 
Push notification through Sats app: provides short and easy to read messages, captures the users attention
SMS: very direct, but can be experienced intrusive 
Email: less likely that the user will bother to read the message, but if it is read, the user may be more likely to absorb the content

Phase 2 - Design 
R6: Selection of design principles - Select appropriate psychological effects (i.e. heuristics and biases). 
The loss aversion: 
Reminding the consequences: 
heuristic systematic model: 

R7: Design of intervention - Design an intervention to induce the desired behaviour based on selected design principles. 
Intervention: participants enroll the intervention. During one month, the users will receive different digital nudges providing them with health related information 

R8: Identification and imitation of successful interventions - Identify relevant examples of persuasive intervent
Similar persuasive interventions: 
Developing persuasive messages for physical activity promotion
 

Phase 3 - Implementation and Evaluation 
R9: Implementation and evaluation of intervention effectiveness - Implement the intervention in the defined technology channel and evaluate it in terms of its effectiveness in achieving the desired user behaviour. 
Effectiveness : The DND method suggest that effectiveness is evaluated, to get a picture of the digital nudges ability to influence the desired user behaviour. However, as this is a qualitative assessment, the measurement collected is more ...

R10: Return to previous phases - If the intervention does not produce the desired effect, repeat the previous phases. 
Further the DND method suggest that if the digital nudge did not successfully achieve the desired behavior, one should return to the design phase and make adjustments.   


\section{Development of Digital Nudges} 
%Explain how the digital nudges that were tested were designed:
For the development of the digital nudge trial DND method (Mirshc et al. 2018) was partly followed. 

Also got inspiration from earlier studies, conducting similar studies. 

This chapter is devoted to describe the design process of the digital nudges that are investigated in this research. 

For the development of the digital nudges investigated in this study, the DND method by Mircsh et al. was used for inspiration and guiding, meaning that is was partly applied / followed loosely. As this is not an evaluation of the DND method itself, but rather to get some guidance on how to decide for what to test. 

This method consist of three phases: analysing, designing and evaluation. the last phase suggest that one should evaluate the design intervention does not fulfill the desired effects, I will not use perform iterative process, as the evaluation part is what my study focuses at. 

\textbf{Timing and Frequency }
Due to the scope of this master thesis / research, timing and frequency, has not been deeply/widely considered when designing these digital nudges.  That is to say, there was chosen three different times during a day and , during one month the participants were supposed to receive 16 nudges in total. Randomly distributed during the 30 days trial.  

\subsection{Digital nudge content / Health information}
%The health information that was presented through the digital nudges were taken from Helsedirektoratet and FHI, who currently has the leading role of informing (but also motivating?) Norwegian residents about physical activity. 
%Henger også sammen med at lack of knowledge er en grunn til barriere?? HVIS det er det da 
The digital nudge intervention was based on health effects and information as it coheres with the objectives of training center that promotes PA witht the context of traningcenter and 

because... passe med treningensenteret/konteksten, med nudgen sin idee... ta gode valg.
And that it supprots good choices. -->As far as consensus research can gather, physical activity is in the best interest of the individual as well as the society. 

The presented knowledge is based on multiple reports and expenses from across the world and has been endorsed by international organizations like WHO and national org. like FHI and Helsedirektoratet. 

%This has been the case for previous studies as well. ... implementing message framing. have also conveyed health related information. 

\subsection{Randomization and broadcast plan}
16 distinct nudges was randomly distributed over a period of 30 days. As timing were not a desired investigation objective, the broadcast of the nudges was randomized, but still within convenient time frames (morning 08-10, noon 12-15, evening 18-20). 

\subsection{Technical implementation}
\begin{comment}Sats har utviklet et eget CMS (content management system) (?) for publisering av nudger gjennom appen.
Sats has developed their own CMS (content management system) for publishing nudges through their app. The nudge is delivered as a push notification on users locked home screen, meaning that there is a connection between X and the operating system. 

Google Firebase provides cloud messaging, which sends notifications and messages to any device (iOS, Android or web based). 
Google Firebase communicates with Device ID (smartphone) 
"Vår serviceplattform"
Operativsystemet på deltakernes enhet 
Google Firebase ble brukt 
Firebase cloud messaging 

%"Selve push meldingen (den som operativsystemet gir bruker, og lever på utsiden av appen); Her bruker vi Google Firebase for å kommunisere med telefonen. Den tar utgangspunkt i Device ID (telefonen) og pusher til Device ID eller IDer (om brukeren er registert med flere devicer). Dette oppfattes som selve push-meldingen. Det er vår serviceplattform som kommuniserer med Google Firebase (og CMSet du kjenner til er det som pusher informasjon inn til vår service plattform). Vår service plattform er forresten ren Microsoft Azure."
\end{comment}

