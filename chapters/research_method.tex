\chapter{Research Method}
%\section{}
%\subsection{}
%\subsubsection{}
This chapter aims to explain the methodological approach and describing the chosen methods for data collection and data analysis. 

\section{Methodology}
%Qualitative data can contribute to increased knowledge about how we should design digital nudges for physical activity promotion. Through a qualitative data (collected during interviews), I want to disclose how we should consider framing when it comes to designing digital nudges in this given context. 

%This study follows an exploratory and qualitative research approach, as we are trying to understand "why" people has certain perceptions about the digital nudge interface. It uses primary data, collected through a qualitative research method (semi-structured interviews).

%Methodological approach: basic, exploratory and qualitative. 

%As the proposed research question for this study - “RQ1” - aims at understanding user's perception and experience of the given digital nudge user interface, (in order to gather insight on how to tailor the digital nudge to be more effective), it would be reasonable to follow a qualitative research approach.

%Interviews has the advantage of being changeable and dynamic, meaning that it allows follow-up questions where unpredictable directions occur. It provides a more detailed dataset than for instance surveys do. It is also possible to converse with several participants at the same time, called focus groups, but there is a risk that participants will be influenced by each other’s responses. 

To answer the addressed research questions, a qualitative and exploratory research approach was applied. In collaboration with SATS, there was conducted a two-week-trial/intervention with digital nudging for physical activity promotion on participants at a fitness center. Followed by data collection of user experience and perceived effect of these digital nudges was collected through interviews. 
We are trying to understand "how" people perceive the digital nudge interface/implementation and "why" (which factors influence the perception) they perceive that way.  

%Data regarding the users experience and perceived effect of these digital nudges was collected through a semi-structured interview. 

\section{Data Collection}
    \subsection{Semi-structured interview}
    %Why did you choose semi-structured interview?
   % Interviews are an flexible way to collect data, as it one can get both the in depth answers and more of a quantitative measurements (such as prior activity frequency before the nudge intervention, type of motivation, etc). @
   As we were mainly looking for qualitative data on the user's experience and perception of the nudges, interviews were an appropriate method to apply for data collection. Interviews are flexible and provided the opportunity to obtain both in-depth knowledge and insight, while also some types of quantitative data (such as prior activity frequency before the nudge intervention, type of motivation, etc). In specific, semi-structured interviews were conducted, as they allow new directions to occur as the participant talks and providing a more open dialog between the interviewer and the interviewee. There was developed a set of predefined topics and questions, which acted as a interview guide to ensure that the same overall topics were reviewed if relevant. The researcher determined to what extent it is useful to follow the guide during the interviews.
    
   We applied open-ended questions to get a detailed and comprehensive picture on the user's experience. Examples of open-ended questions: What motivates you to stay physically active, how do you deal with push notifications, how do you perceive the message of the nudges, why you use the Sats app and what do you use it for, what nudges do you remember what effect did the nudge have on you, etc. It was crucial to ask appropriate follow-up questions where necessary to capture important directions amongst the participants.   
\begin{comment}    
\subsection{Participants and Recruitment}
\textbf{Target population and study sample}

\textbf{Sample Size} 
 We were able to conduct 22 interviews, which we considered suitable, as this is an exploratory study of a certain implementation of a digital nudge interface, which aims to investigate users perception and experience. 

\textbf{Sampling technique /Recruitment}
%Sampling means how the data sources was chosen whilst recruitment means how it was actually performed(?)
    
    The study is limited to investigating how individuals with ability and motivation for physical activity (apparently), experience and perceive the digital nudges for physical activity. Hence, the sampling was non-randomized.
    
     %The easiest way to get in touch with the target population (individuals who have expressed ability and motivation for physical activity through membership in the training center) was to attend on some training centers. 
     The recruitment was done by standing at various SATS training centers in Oslo and actively recruiting people for the digital nudge trial / intervention. Information sheets was distributed, and consents where collected. In addition it was made sure that everyone signing up to participate had enabled push notifications through their app. In total, 105 participants registered to be a part of the study. Even though everyone agreed to take part in the following interview, we knew from experience from other research, that the actual participation rate is not very high. For that reason, we wanted to recruit as many participants as possible over a time period. 
        
    As the study followed an exploratory approach, it was also nice to have a good selection of participants, if it should be necessary to group/sort/filter them more.
    
%\cite{online_chapter_nodate} 
\end{comment} 
 
    \subsection{Data Collection Procedure}
A step by step description of data collection procedures will follow. 
    
        \subsubsection{Preparations}
        %nsd, samtykkeskjema, definere intervju, velge ut objeter, framdriftsplan 
        As this project deals with personal and health-related data, it was reported to NSD, Norwegian Centre of Research Data, where it was approved. In collaboration with NSD there was created an “informed consent form” (see Appendix B) with important information about the research (such as data processing and data storage, and the participants rights), that the participants had to read in order to give their consent to participate in the study. 
        
        \subsubsection{Recruitment}
        %Skal den plasseres her?
        The recruitment was done by standing at various Sats training centers in Oslo and actively recruiting people for the digital nudge trial. Information sheets and consent forms was distributed to people who expressed interest in participating in the study. In addition it was made sure that everyone signing up to participate had enabled push notifications through their app. In total, 105 participants registered to be a part of the study. Even though everyone agreed to take part in the following interview, we knew from experience from other research, that the actual participation rate is not very high. For that reason, we wanted to recruit as many participants as possible over a time period. 
        
        %As the study followed an exploratory approach, it was also nice to have a good selection of participants, if it should be necessary to group/sort/filter them more.
        
        \subsubsection{Interview Guide}
	As mentioned, an interview guide was designed to ensure that roughly the same themes were touched upon and addressed in each interview when/if relevant. Based on the literature, we had some ideas about what we could find in the data, and these were some of the topics in the interview guide, for instance mapping the user's relationship to physical activity (where motivation, goals and activity habits were asked), mapping experience with such technology (user engagement, attitude, experience with similar technology) and mapping the experience and perception (credibility, relevance, etc)
        
        \subsubsection{Pilot Interview}
Before starting the interview process, a pilot test of the interview guide was conducted to ensure that the right questions were asked in the right manner, and that  it managed to get useful / promising data from the participants. - should ensure that all questions are understandable and clear to the participants (Reference: Research Methods in Human-Computer Interaction, 2nd Edition, 2017). Questions and topics that appeared to be irrelevant was removed, and biased or ambiguous questions were changed to a more neutral and clear manner. In addition, the pilot testing also gave an approximately estimation of how long time one interview would take; 45 min. The participants were informed that the interview would take 30-60 min (and not 20 minutes as it was states in the information sheet).
        The pilot interview also made me more confident and comfortable in the new role as the interviewer.
        
        \subsubsection{Interview Execution}
        In total 22 interviews were conducted. The interviews were conducted within April 2020. The interview was executed using different video calling tools (WhatsApp or FaceTime). Each interview took between 30 and 60 minutes.  
        The interview started with a refresher/introduction on what the study is about and the purpose of it. Participants were informed of their rights to withdraw during the interview. 
        %For the execution of interviews there are several considerations to be aware of. To get the most out of each interview object, one should ensure that the participant feel comfortable, safe and relaxed in the situation. 
\section{Data Analysis}
    \subsection{Thematic Analysis}
    %What is thematic analysis? And why did you choose it? maks 2 sider
    When following a thematic analysis approach several choices needs to be considered and discussed, according to Braun and Clark (Reference: Using Thematic Analysis in Psychology, 2006). 
    
    A latent approach was used to identify themes in the dataset, i.e interpretive.  
    
\begin{itemize}
\item     Semantic vs latent
\end{itemize}
    A thematic analysis means... 
    
    \subsection{Data Analysis Procedure}
    There are usually six stages of a thematic analysis: 1) familiarization with data, 2) coding, 3) generating themes, 4)Reviewing themes, 5) defining and naming themes and 6) produce the report.
    
    The following paragraph will present the phases of the thematic analysis that was conducted. 
    
    \subsubsection{Preparations}
    The verbal data was transcribed during the interview due to the limitations and lack of tools and resources. Due to this, the process around generating follow-up questions was a bit delayed and weakened. The risk about transcribing simultaneously with interviewing, is that it can be easy to miss things, however this was the best alternative pga økonomiske svakheter / in this case. Directly after each interview, there was spent a good deal of time going through the interview to correct the typos and make the sentences complete. After all interviews were conducted and transcripted, the above-mentioned steps of thematic analysis was followed // used as inspiration, however, some adjustments were made. A description follows. 

    \subsubsection{Defining user groups}
    Generating user groups based on different characteristics in the data set: Various groupings were generated amongst the participants. As there was great variation among participants (representative selection), it was interesting to define some groups (consisting of individuals with the same variables / characteristics) to see if there was any correlation between their opinions, experiences and perceptions. The participants were grouped on variables such as: prior activity frequency, type of motivation (intrinsic or extrinsic), age.
    Age: The variation in the age of the participants was slightly skewed: more young than older people. In the age group of "20-30 years", there were 16 participants, whilst in the age group "30+ years" there were only 6. Therefore it is not possible to generalize findings around the age group, but we have nevertheless taken that into account.
    Type of motivation:
    Prior level: 
    
    \subsubsection{1. Familiarization}
    The first step of a thematic analysis is all about getting familiar with the data set. The interviews was read and re-read several times, ensuring an overview of the data set as a whole, and as independent materials, before starting the coding. During this phase, notes were taken in order to have some ideas about what the data provides and why it is interesting before coding starts. 
    
    \subsubsection{2. Coding}
    The data was coded manually as this allows for a more focus on the creative, open minded researcher role... Phrases from the raw data (interview data) which described an idea, opinion, experience or feeling of a participant were highlighted and marked with labels that explained what the data represents. Different colors were used to sort and categorize the highlighted phrases into meaningful codes. After the coding phase there was X number of codes. 
    
    Which coding approach: inductive or deductive? 
    
    \subsubsection{3. Generating themes}
    When it comes to identifying themes among the codes, it can be done in two different ways. Latent (interpretive), which means that one has to interpret what one thinks the participant means by what it says. Semtanktik (explicit), which means to adhere to exactly what they say. 
    
    A latent approach was followed as ...
    satt sammen koder til midlertidige tema, nøye gjennom bla bla 

    \subsubsection{4. Reviewing themes}
    \subsubsection{5. Defining and naming themes}
    
    \subsubsection{6. Presenting results}
    The presentation of qualitative data is varied. 
    Finding a meaningful way to present and describe the findings. 

Such analyzes, where interpretation of textual data is carried out, meet special considerations of validity (Reference: Research Methods in Human-Computer Interaction, 2nd Edition, Chapter 8. part 8.10.2.) If the study aims to present generalized findings, several researchers should be included in the analysis process. This to avoid the subjectivity that naturally arises when only one person is responsible for the analysis. The homogeneity and consensus between the independent analyzes will then be assessed, and in this way, inter-rater reliability can enhance the analysis.

\section{Limitations}
%bare en forsker - svekker validitet og realibilitet
%testgrunnlaget for intervjuene var svekkes ettersom kun 8 av 16 nudger ble sendt ut til deltakerne

      %Hører til Discussion eller Limitations? As the interviews were conducted using technology tools it could have influenced the data collection. For some people, having to use technology in order to participate might be just another barrier to actually participate in the interview, whilst for other see it as lower threshold participating. Hence, the technology tool for video calling was made on participant's premise
     
\begin{itemize}
\item Validity 
\end{itemize}


