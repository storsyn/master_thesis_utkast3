\chapter{Research Method}
%\section{}
%\subsection{}
%\subsubsection{}
This chapter aims at explaining the methodological approach and describing the chosen methods for data collection and data analysis. 

\section{Methodology}
%Qualitative data can contribute to increased knowledge about how we should design digital nudges for physical activity promotion. Through a qualitative data (collected during interviews), I want to disclose how we should consider framing when it comes to designing digital nudges in this given context. 

%This study follows an exploratory and qualitative research approach, as we are trying to understand "why" people has certain perceptions about the digital nudge interface. It uses primary data, collected through a qualitative research method (semi-structured interviews).

%Methodological approach: basic, exploratory and qualitative. 

%As the proposed research question for this study - “RQ1” - aims at understanding user's perception and experience of the given digital nudge user interface, (in order to gather insight on how to tailor the digital nudge to be more effective), it would be reasonable to follow a qualitative research approach.

%Interviews has the advantage of being changeable and dynamic, meaning that it allows follow-up questions where unpredictable directions occur. It provides a more detailed dataset than for instance surveys do. It is also possible to converse with several participants at the same time, called focus groups, but there is a risk that participants will be influenced by each other’s responses. 

To answer the addressed research questions, a qualitative and exploratory research approach was applied. In collaboration with Sats, a two-week intervention with digital nudging for physical activity promotion on participants at a fitness center was conducted. Thereafter, data collection of user experiences and perceived effect of these digital nudges were collected through in-depth interviews. We are trying to understand in what manner and to what degree people perceive the digital nudge interface and why it is perceived that way. By analysing qualitative data about users opinions, attitudes, experiences, perceptions, etc, regarding the given digital nudge implementation, we aim to disclose how we should consider framing and other aspects (delivery method), when it comes to designing and tailoring digital nudges for this user group and context. 

%liste forskningsspørsmålene og forklare hvordan de skal bli besvart og hvilke data som er interresante for dem?

This research is exploratory in nature as it is the first study regarding digital nudging for this user group and context. Findings from this study will therefore not be able to answer the big questions completely, but may instead provide guidelines for further research. 

Commonly used methods to investigate digital nudge interventions, message framing and physical activity promotion, etc 
%Data regarding the users experience and perceived effect of these digital nudges was collected through a semi-structured interview. 

\section{Data Collection}
The following sections will present the method for data collection, in addition to a description of the procedures. 
    \subsection{Semi-structured interview}
    %Why did you choose semi-structured interview?
   % Interviews are an flexible way to collect data, as it one can get both the in depth answers and more of a quantitative measurements (such as prior activity frequency before the nudge intervention, type of motivation, etc). @
    Emphasis was put on collecting qualitative data hence interviews was found to be the most adequate method to apply for data collection. Interviews were flexible and provided the opportunity to obtain both in-depth knowledge and insight, and at the same time it opened for collecting and correlating certain types of data quantitatively (such as prior activity frequency before the nudge intervention,  type of motivation, age, gender, etc). More specifically, semi-structured interviews were conducted, as they allowed new directions to occur during the sessions and thus provided a more open dialog between the interviewer and the interviewee. A set of predefined topics and questions was developed, which acted as a interview guide to ensure that the same overall topics were addressed together with each and every participant. During the interview, it was determined and exercised to what extent the predefined subordinate questions were useful. 

    Open-ended questions were asked to get a honest, unbiased and comprehensive picture on the user’s experience. Examples of open-ended questions are: what motivates you to stay physically active, how do you deal with push notifications, how do you perceive the message of the nudges, why do you use the Sats app and what do you use it for, what nudges do you remember, what effect did the nudge have on you?, etc. It was crucial to ask appropriate follow-up questions when necessary to capture important directions amongst the participants, i.e. delineating and sorting based on previous answers and creating an open atmosphere.
  
\begin{comment}    
\subsection{Participants and Recruitment}
\textbf{Target population and study sample}

\textbf{Sample Size} 
 We were able to conduct 22 interviews, which we considered suitable, as this is an exploratory study of a certain implementation of a digital nudge interface, which aims to investigate users perception and experience. 

\textbf{Sampling technique /Recruitment}
%Sampling means how the data sources was chosen whilst recruitment means how it was actually performed(?)
    
    The study is limited to investigating how individuals with ability and motivation for physical activity (apparently), experience and perceive the digital nudges for physical activity. Hence, the sampling was non-randomized.
    
     %The easiest way to get in touch with the target population (individuals who have expressed ability and motivation for physical activity through membership in the training center) was to attend on some training centers. 
     The recruitment was done by standing at various SATS training centers in Oslo and actively recruiting people for the digital nudge trial / intervention. Information sheets was distributed, and consents where collected. In addition it was made sure that everyone signing up to participate had enabled push notifications through their app. In total, 105 participants registered to be a part of the study. Even though everyone agreed to take part in the following interview, we knew from experience from other research, that the actual participation rate is not very high. For that reason, we wanted to recruit as many participants as possible over a time period. 
        
    As the study followed an exploratory approach, it was also nice to have a good selection of participants, if it should be necessary to group/sort/filter them more.
    
%\cite{online_chapter_nodate} 
\end{comment} 
 
\subsection{Data Collection Procedure}
A (chronological) step by step description of data collection procedures will follow. 
        \subsubsection{Preparations}
        %nsd, samtykkeskjema, definere intervju, velge ut objeter, framdriftsplan 
        As this project dealt with personal and health-related data, it was reported to NSD, Norwegian Centre of Research Data, by whom it was approved. In collaboration with NSD an “informed consent form” was created (see Appendix B) with important information about the research (such as data processing and data storage, and the participants rights), that the participants had to read and sign in order to give their consent to participate. 
        
        \subsubsection{Recruitment}
        %Skal den plasseres her?
        The recruitment was done on location at various Sats training centers in Oslo, by actively contacting members passing by. Information sheets and consent forms were distributed to people who expressed interest in participating in the study. In addition it was assured that everyone signing up to participate had enabled push notifications through the Sats app. In total, 105 participants registered to be a part of the study. Even though everyone agreed to take part in the following interview, it is known that the actual participation is significantly lower due to different circumstances. Thus it was a target to recruit  as  many participants as possible over a limited period of time period.
        
        %As the study followed an exploratory approach, it was also nice to have a good selection of participants, if it should be necessary to group/sort/filter them more.
        
        \subsubsection{Interview Guide}
	    As mentioned, an interview guide was produced to ensure that the same themes were touched upon and addressed in each interview as found relevant. Based on the literature, we had some ideas about what we could find in the data, and these were some of the topics in the interview guide, for instance mapping the user’s relationship to physical activity (where questions concerning motivation, goals and activity habits were asked), mapping experience with similar technology (user engagement, attitude, prior experience with similar technology) and mapping the experience and perception (credibility, relevance, etc).  The interview guide was peer-reviewed in order to remove bias and preserve stringency. Due to unforeseen circumstances related to the lateral extent of the intervention, some adaptations and adjustments to the interview guide became necessary. 
        
        \subsubsection{Pilot Interview}
        Before starting the interview process, a pilot test of the interview guide was conducted to ensure that the right questions were asked in the right manner, and that it provided useful data from the participants.  Furthermore, the pilot interview ensured that all questions were comprehensible and clear to  the  participants (Reference:  Research Methods in Human-Computer Interaction, 2nd Edition, 2017).  Questions and topics that appeared to be incomprehensible and/or irrelevant were removed. Biased or ambiguous questions were changed to a more neutral and clearer phrasing. In addition, the pilot testing also gave an approximate estimate as to how long time one interview would endure; 45 min. Thus, one finding from the pilot interview was that it was deemed necessary to extend the time frame, i.e. from the intended 20 minutes as stated in the information consent, to 30-60 minutes per interview. This was informed about prior to the actual interviews.
        
        In general terms, it is very useful with a pilot interview because it may strengthen the role of the interviewer in terms of confidence and control (reference), especially for novice researchers unacquainted to such settings. 
        
        \subsubsection{Interview Execution}
        In total 22 distinct interviews were conducted, within April 2020. The interview were executed using different video conferencing tools (such as WhatsApp or FaceTime), dependent on interviewee preferences. Each interview took between 30 and 60 minutes like set forth. The interview started with a refresher of what the study was about and its purpose. Participants were informed about their right to withdraw during the interview. Efforts were made to foster a open dialog and to make the interviewee comfortable during the session. 
 
        Like mentioned, some adaptations were needed to collect the wanted data. Some of the nudges intended for broadcasting had to be presented during the interview session to provide a meaningful opinion on the topic. Both WhatsApp and FaceTime could facilitate for sending text and thus mimicking push notifications. The participant was presented with some of the digital nudges during the interview for the questions that specifically dealt with the perception of those.
        %HJELP PAPPA
        The interviews were not recorded due to set restrictions and delineations. The interviews were documented by real-time transcriptions. 

        %For the execution of interviews there are several considerations to be aware of. To get the most out of each interview object, one should ensure that the participant feel comfortable, safe and relaxed in the situation. 
\section{Data Analysis}
In the following sections the method for data analysis will be presented, and further expanded on by describing the different phases of the process.  
    \subsection{Thematic Analysis}
    %What is thematic analysis? And why did you choose it? maks 2 sider
    When following a thematic analysis approach several choices need to be considered and discussed, according to Braun and Clark (Reference: Using Thematic Analysis in Psychology, 2006). 
    
    A latent approach was used to identify themes in the dataset, i.e interpretive.  
    
\begin{itemize}
\item     Semantic vs latent
\end{itemize}
    A thematic analysis means... 
    
    \subsection{Data Analysis Procedure}
    There are usually six stages of a thematic analysis: 1) familiarization with data, 2) coding, 3) generating themes, 4)Reviewing themes, 5) defining and naming themes and 6) produce the report.
    
    The following paragraph will present the phases of the thematic analysis that was conducted. 
    
    \subsubsection{Preparations}
    As mentioned in section about interview execution, the verbal data was transcribed during the interview. There was X number of words of transcripted data. 
    %Due to this, the process around generating follow-up questions was a bit delayed and weakened. The risk about transcribing simultaneously with interviewing, is that it can be easy to miss things, however this was the best alternative pga økonomiske svakheter / in this case. 
    Directly after each interview, there was spent a good deal of time going through the interview to correct the typos and make the sentences complete. 
    %After all interviews were conducted and transcripted, the above-mentioned steps of thematic analysis was followed // used as inspiration, however, some adjustments were made. A description follows. 

    \subsubsection{Defining user groups}
    Generating user groups based on different characteristics in the data set: Various groupings were generated amongst the participants. As there was great variation among participants (representative selection), it was interesting to define some groups (consisting of individuals with the same variables / characteristics) to see if there was any correlation between their opinions, experiences and perceptions. The participants were grouped on variables such as: prior activity frequency, type of motivation (intrinsic or extrinsic), age.
    
\begin{itemize}
\item Age: The variation in the age of the participants was slightly skewed: more young than older people. In the age group of "20-30 years", there were 16 participants, whilst in the age group "30+ years" there were only 6. Therefore it is not possible to generalize findings around the age group, but we have nevertheless taken that into account.
\item     Type of motivation: intrinsic (motivated by internal/personal reward) or extrinsic (motivated by external reward)
\item     Prior level of physical activity: the participants were categorised into three different levels of physical activity on average during a week; engaging in 2 or less sessions of physical activity , 3-4 sessions of physical activity, 5+ sessions of physical activity. 
\end{itemize}
    
    \subsubsection{1. Familiarization}
    The first step of a thematic analysis is all about getting familiar with the data set. The interviews was read and re-read several times, ensuring an overview of the data set as a whole, and as independent materials, before starting the coding. During this phase, notes were taken in order to have some ideas about what the data provides and why it is interesting before coding starts.
    \subsubsection{2. Coding}
    The data was coded manually as this allows for a more focus on the creative, open minded researcher role... 
    %Inductive or deductive? 
   Coding can be performed using different approaches; inductive (data-driven) and deductive (theory-driven). As the research questions for this study was two fold (both open overall experience/perception question, and more specific how message framing did affect the experience), we wanted to perform a hybrid approach of inductive and deductive. This allowed us to find out about new things we didn't know about on beforehand, while also expanding and comparing to the existing theory. 
   Firstly, we coded with a inductive approach, as we wanted to explore what you found out without being influenced by theory.  This proved to be challenging. At the same time, we needed data to answer the specific research questions and hence we applied deductive coding as well. 
    
    Codes where made going through the raw data (interview data) line by line. Phrases from the which described an idea, opinion, experience or feeling of a participant were highlighted and marked with labels that explained what the data represents. Different colors were used to sort and categorize the highlighted phrases into meaningful codes. Everything that was perceived as interesting, surprising, what the user himself defined as important and everything that was relevant to the research questions, were coded.  After the coding phase there was in total 45++ generated codes.  
    %Which coding approach: inductive or deductive?
    \subsubsection{3. Generating themes}
During this phase, the codes were carefully analyzed, and various themes were identified to embrace and sort correlated codes. In other words, relations between codes and themes were mapped. To get an overview of the different codes and which themes they belonged to, it was made some visualisations, including a mind map (figure x). As a result, potential overarching and sub-themes were generated. Those codes that did not fit into the generated themes was put in an temporary theme called "other". Even though some themes seemed less relevant than others, we still kept them for the next phase. 

    \subsubsection{4. Reviewing themes}
The potential themes was then reviewed 
Her ble det undersøkt om temaene faktisk hadde sterke nok argumenter til å utgjøre et eget tema, eller om de kunne slås sammen med andre temaer. 
Først så vi gjennom kodene som tilhørte et tema. Så så vi gjennom kodene. Nye tema ble dannet, noen ble sammensmeltet eller splittet opp, og noen ble forkastet. 

    \subsubsection{5. Defining and naming themes}
Analyse for å spesifisere hva hvert tema innholder og dreier seg om

klare definisjoner, navngvning, 
    
    \subsubsection{6. Presenting results}
    The presentation of qualitative data is varied. og vhilke data skal presenteres og på hvilken måte. finne ut hvilke eksempler som skal brukes. Kan godt være man har mer data enn det som blir nødvendig og relevant å legge frem for å svare på forskningsspørmålene. 
    Finding a meaningful way to present and describe the findings. 

\section{Ethical Considerations}
Including human subjects in research is faced with various consideration, especially ethical. As mentioned, a informed consent was constructed to safeguard the participants and meet the ethical requirements. 

%As this study handle data regarding users relationship to physical activity (which is considered as health related data) it can be associated with special vulnerability(?). 

\section{Validity and Reliability}

An important aspect of ensuring quality over research projects is the ability to achieve validity and reliability. As this study perform an analysis where interpretation of textual data is carried out, it meets special considerations regarding validity (Reference: Research Methods in Human-Computer Interaction, 2nd Edition, Chapter 8. part 8.10.2.) Several researchers should be included in the analysis process, to avoid the subjectivity that naturally arises when only one person is responsible for the analysis. However, as this study does not aim for presenting generalized findings (and due to other delineations), it eases the demand for several researchers. 

By performing a pilot test of the interview guide ahead of the actual data collection, the reliability and validity of the study is strengthen.


%Validity (trustworthiness) - accuracy of measure (sampling method, sample set, size)
%Reliability (consistency) - consistency of a measure




