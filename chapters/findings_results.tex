\chapter{Findings}

%\textbf{Present your findings.}

\subsubsection{Memorability}
Interesting: everyone claims to have read the messages but only half (?) remembers anything of the information presented. This indicates that system 2 (automatic mind) seems to process the information they are reading. 

%Several of the participants mentions that they want tailored content, for example based on earlier activity history, feedback, challenges etc, but it goes more in the direction of motivation training apps, but is of course a important aspect of persuasive app in whole, but it is hard to say where digital nudging has responsibility(?)

\subsubsection{Novel vs familiar information}
It seems that topics that present new information to the participant are more interesting / memorable: the nudge about cancer and cancer types

\subsubsection{Timing}
Timing: those participants who have a strong memory of one or more of the nudges they have received, also explained and described which context (where, when and personal emotions) they were in when they received it. This may indicate that they remember it because of the situation they were in when they received the nudge.
\begin{itemize}
\item Hangover: the person was waking up after a night out, feeling hangover and received a nudge, he ended up taking a longer walk with the dog and planning workout the day after
\item After a workout: the person was feeling that things were kjipt due to the corona situation blussing up, and received a nudge on the way home from a workout which made her feel happiness, achievement and unity 
\item Out of ordinary / extreme work situation: 
\end{itemize}

Such findings boost the need / interest to research timing and digital nudging in particular. which supports suggestions from other studies. 

\subsubsection{Health Topics}
Nærliggende tema / tema man kan kjenne seg igjen er mer interresante. Men hva defineres som er nærliggende tema? Deltakere sa: "fordi jeg kjenner en med kreft", "fordi min far har slitt med det", "fordi jeg har hatt en kneskade", "fordi jeg har en leddsykdom", etc. 

\subsubsection{Age}
Alder? Naturlig nok ser vi en forskjell mellom eldre og unge; hvor eldre gjerne føler seg veldig klar over helsefaktaene fra før, mens noen av de unge hører det for første gang. 
Over the age of fifty: experienced the nudges as preachy / belærende / moralizing. Why? Maybe because they lived longer and have learned more. 

\subsubsection{User level}
Brukernivå av appen: hvis man er veldig engasjert i appen vil det være mer naturlig å bry seg om varslingene?

\subsubsection{Gender}
Kjønn?

\subsubsection{Work}
Yrke / arbeidssituasjon: several of the participants mentioned their occupation and work situation

\subsubsection{Activity level}
Activity level: some (total number ?) participants where on a low activity level, recently started training (?), exercise once a week, and they seemed not to be interested in physical activity and training, but worked out because of friends and weight control. They were... 

\subsubsection{Credibility}
Credibility : it was a case where the participant was very critical of this information and was unsure if he could trust it.

\subsubsection{Attitude}
Attitude: even though most of the participants had the 

\subsubsection{Message framing}
Message framing: most of the participants would prefer gain-framed messages as they experienced them more positive and motivating. 
We were supposed to test whether or not, the participants actually noticed the difference between the framing of the messages. It turned out that most participants noticed the content and health category before the message framing, which could be a sign that the content is more important than how the message is framed. Also, some participants did not prefer gain or loss over the other, and behaved similarly to them, which is a counter-argument to the fact that message framing is as important as other studies claim.

When the participants where asked which of the nudges they would prefer, a majority of the participants (20 of 22?) answered the nudges under the category of gain-framed messages. 
As trial period was not completed due to corona, the basis for asking about message framing was a bit weak. Therefore, I presented all the nudges in a randomly manner to the participant during the interview, and which was more effective on them. I did not tell them anything about the topics and message framing. The participants got some time to read through it. A majority of the participants pointed out only gain-framed messages.  

\subsubsection{User engagement}
 As user engagement is showed to have a correlation with effectiveness (????) of a certain method in persuasive technology, it was interesting to look at. Most of the participants told that they read the push notification and thought nothing more about it. A couple told that they pressed into the app after receiving the notification
 
Oppleves nudgene anderledes mellom deltakere med ulik aktivetesfrekvens? 
Hvordan påvirker aktivietsfrekvens den opplevde effekten av digital nudging?
Particpants with lower than 2 training 
Is there any correlation between activity frequency and perceived effect of nudging?

\subsubsection{Evoked Emotions}
During the interviews, I discovered various emotions were evoked by the nudges. Some participants expressed specific emotions regarding the nudges. 
\begin{itemize}
\item Provoking
\item Funny
\item Yelled at
\item Happy
\item Oppgitt
\end{itemize}

One participant (in the lowest level of physical activity frequency) expressed that he felt demotivated by the loss-framed messages. 

When and why was the nudges effective?
Some expressed that if the information would be new to them and they could learn anything from it, it would be more interesting and relevant and thus may have a bigger impact on them.
Other expressed that they liked the fact that it was common knowledge, where most of the information was already known at some level, because everyone should know about it. 
Noen mente de ble minnet på hovedgrunnene til å drive med fysisk aktvitet, noe som var fint, for det var ikke alltid denne informasjonen som fikk de til å dra på trening (det kunne heller være dårlig samvittighet og sånne negative følelser)

Nudgene som inneholdt informasjon om hverdagslige tema som stress, ble mer satt pris på. 

\begin{table}[ht]
\begin{center}
\begin{tabular}{ |  m{10em} | m{1.5cm}| m{1.5cm} | m{1.5cm} |} 
\hline
  &  \textbf{Gain} & \textbf{Loss} & \textbf{None} \\ 
\hline
\textbf{Extrinsic motivation} & 10 & 2 & 1  \\ 
\hline
\textbf{Intrinsic motivation} & 5 & 1 & 3  \\ 
\hline
\end{tabular}
\caption{\label{tab:table-name} Shows how many participants preferred the different wording of the message.}
\end{center}
\end{table}

heidahvordan går det

\begin{comment}
\begin{table}[ht]
\begin{center}
\begin{tabular}{ | m{2cm} | m{6cm} | m{6cm} | } 
\hline
  &  \textbf{Gain} & \textbf{Loss}  \\ 
\hline
\textbf{Positive statement} & feels like it rewards me for exercising & powerful, stronger, serious \\ 
\hline
\textbf{Negative statement} & & patronizing, provoking, insulting \\ 
\hline
\end{tabular}
\caption{\label{tab:table-name}This table shows positive and negative statements related to gain and loss.}
\end{center}
\end{table}
\end{comment}









