\chapter{Discussion}
This chapter expand on and discuss the presented findings, and compare it with findings of the previous studies reviewed in chapter 2. 

\section{Users experience and perception}
\subsection{Message Framing}
Many earlier studies have pinpointed how gain-framed messages often ... in health contexts. 

\subsection{Timing / Context}

\section{Limitations}
  %Utfordring med validitet
Due to time constrictions, corona and ethics of doing research we used convenience sampling. As we imposed convenience sampling for this study, the 

\subsection{Validity}

\subsection{Sample selection - Representativity}
Among those who signed up as participants in the study, the selection was representative. But due to the time constrictions (lost time due to corona) we had to make interviews with the ones available first. We did not have time to sort of age and gender, even though we had the opportunity for it in regards of number of participants and the variation in age and gender. The time limitations was one of the consequences of corona pandemic. 

The intention to include training members was that among this group / selection of people there will always be someone who trains a lot and someone that does not, meaning / in other words; it was representative for a large part of Norway's population. However, all members of training center share / has one ting in common; the motivation / intention to engage in some form of physical activity, because they already became a member (invest money in their own health). When we consider Fogg's behaviour model, we see that motivation is one of three fundamental elements that needs to be in place for a behaviour to occur. Ability and prompts are the other two. 

\section{Additional investigations triggered by Corona situation}
We know that nudging already has been implemented in a number of areas of government in different countries, also digital nudging. For the situation that hit the world this year, it is particularly interesting to look at how digital nudging can be utilized for better choices in such a situation, for example to contribute to the stopping of infection. In the matter of fact, a quite rare example of nudging appeared during the corona crises in Norway.  

The SMS from FHI could remind more as a government order, than a nudge. 
Credibility : myndighetspålegg 
SMS vs push notification or other channels : The SMS from FHI was experienced as more direct communication and , therefore also more , is perceived more as a citizen's duty, oppfattes mer som borgerplikt og myndighetspålegg , dette er nudge i en ekstrem form, 
One can look at this as nudging in extreme form; no choices are made by the user,
FHI dosent have any other channel to communication to everyone at the same time, 

Kunne vært interresant å tatt dette tema og denne situasjonen / konteksten videre innenfor forksningen for digital nudging. Samfunnsikkerhet, på samme måte som cyber sikkerhet og trafikksikkerhet. (infraksturertur, tilgan gpå mat o varer, logistikk, trafikk?)

%Future work eller discussion? 
Even though this way of nudging (informational health nudges) immediately invites the reflective mind, it would be interesting to see how such intervention would lead to over time.  For future work, it would also be interesting to look at different types and ways of implementation. In particular looking at those who talks to the automatic mind. 


