\chapter{Introduction} 
%\section{}
%\subsection{}
%\subsubsection{}

%Introduce your topic.• State of the world…• The big BUT…• Therefore, we did…• The key findings are…• The contributions of this work are…

\section{Statement of problems}
State of the world: is characterized by technology and inactivity... 

Recommendations from policymakers are not sufficient motivating factors to get people into physical activity. On a global plan the lack of physical activity is still a problem. And it seems that to fill in this lack is very difficult. Creating such changes is vital. 

Most people are aware of the many benefits physical activity gives us. In Norway, we learn about why physical activity is good for us in the childhood, and it continues through school, education and work. However, health statistics indicate that education is not good enough to ensure people to stay physical active on a regular basis. Numbers from FHI, shows that only 30 percent of Norway's residents fulfill the minimum recommendations of physical activity. 
%\cite{noauthor_fysisk_nodate}

(Reference: https://www.fhi.no/nettpub/ncd/fysisk-aktivitet/voksne/) Through a questionnaire where participants where supposed to self-report their physical activity, showed that about 30 percent of adults (20-85 years) in Norway are inactive. However, the results that have been measured with an accelerometer are 70 percent, these data are of higher quality than self-reported data (from experience one can gather that individuals have tendency to exaggerate opinions and behaviour that would put them in a better light). The large gap between self-reported and measured physical activity may imply that people are less active than they appear or believe in. (It should be mentioned that the number of participants for the self-reporting survey was low, which gives uncertainty about the data being representative of the entire population.) (Reference: Https://www.fhi.no/nettpub/ncd/fysisk-aktivitet/voksne/)

Norway is one of many countries which has committed to WHO's global goal of reducing inactivity. The concrete goal for 2025, it to reduce it by 10 percent. (Reference: https://www.fhi.no/nyheter/2018/ny-handlingsplan-skal-fa-folk-opp-av-sofaen/). The Norwegian Institute of Public Health (FHI), a Norwegian governmental agency subject to the Ministry of Health and Care Services, seeks an overview of best practices and initiatives that contribute to physical activity among Norway's population groups...

Although the advice from the Directorate of Health is available in rich detail online, they still have the potential to become more accessible and embedded in everyday tools and services, such as technological devices. However, it is not that simple. Making education and information more available and persuasive will not alone drive the people at the wrong side of the scale into becoming enough physical active. The psychology of human behavior and absence from behavior is more complex than our ignorance (uvitenhet). Visiting the field of behavior science, B.J Fogg insist that there are several factors that needs to be present at the same time in order to achieve a target behaviour; motivation, ability and trigger/prompt, is supposed to be the key to success. 

Take the scenario of a training center: everyone seemingly has the ability and some kind/level of motivation to be physical active as they are paying a monthly price. Now, how do we trigger or prompt them?  This is where technology comes in. It would be interesting to embed the knowledge from these actors (FHI/Helsedirektoratet) into a digital context for people who motivation and ability is somewhat present. There are several ways to implement such an intervention. 
 
This trigger factor is where we can utilize technology and user interfaces. Further Fogg describe three different variations of the trigger; facilitator, spark and signal. The latter is best to use when motivation and ability is present, (where considerations regarding tailoring, frequency and timing).

%There is a need to make this information more available and persuasive. 
%Therefore it is interesting to embed the knowledge from these actors into everyday technology (for instance like an gym app). 

So what kind of nudge/trigger/prompt do people need to change this behavior? This debate does not only revolve around making totally inactive people to become active, but also making those who are already active to keep it on a regular basis. 

%For it to be sustainable, it should also be taken into account that the changes must be long lasting and preferably permanent.

This study will transfer theories from the field of psychology into digital nudging, in order to gain insight on users perception and experience of the given implementation of digital nudging, which again can contribute to establishing if this is an effective method of doing it. 

We are investigating three factors; 1) how wording of a message (message framing) , 2) how the chosen channel, push notification, affects it and 3) how the content (?)... can affect the perceived effectiveness. 

%There has been done several studies regarding the impact of communication, framing and other. For instance in context of persuasive systems, health communication system, behaviour change system, etc. It can remind and sometimes overlap the concept of digital nudging. However, to the best of my knowledge this is the first experiment including framing for digital nudging in specific. 

%Intro: verdens bilde: mer teknologi + mindre aktivitet
%Snu det om, bruke teknologi til å få folk mer aktive
%Tidligere forskning, persuasive design og tech etc... hva har blitt gjort ang fysisk aktivitet
%Vi har forstått at vi må ta mer fra pyskologien for å knekke design koden å kunne lage de beeste profuktene
%det er dette hci mye bunner i
%emrging trends: digital nudging, som faktisk stammer fra behavoirual ecomics..
%litt hva det er og som er funnet
%her er også spørsmålet: how to nudge?
%må låne ting fra andre fagfelt, gjøre tester av ulike mekansimer og ways of implementation, for å kartlegge for hvilke situasjoenr man bør benytte seg av hvilke type digital nudging. 

%Denne oppgave evaluerer en implentasjon av digital nudging for å fremme fysisk aktvitiet i en treningsapp kontekst. Dette gjør vi i forsøk på å komme med innsikt og bidra med funn til det store overordnetet spørsmålet "How to nudge". 

\section{Research Questions}
\begin{itemize}
\item RQ1: How are digital nudges based on health information, experienced /or/ perceived within a persuasive gym app context /or/ for gym members?
\end{itemize}
\begin{itemize}
\item RQ1a: How does message framing affect the user perception of the digital nudge?
\item RQ1b: How are push notifications experienced as the channel for broadcasting digital nudges?
\item RQ1c: Which other factors influence the perceived effect of this implementation of digital nudging?
\item RQ2: Which personal characteristics should be considered when tailoring for this user group and context?
\end{itemize}

These questions are asked in order to answer the bigger problem statement: how to nudge and what to consider when tailoring digital nudges for physical activity promotion? Through this study I also aim at gaining insight into which factors should be considered when it comes to tailoring of digital nudges. 

%\section{Hypothesis}
%\begin{itemize}
%\item H1: Message framing will impact the user perception of digital nudges.
%\item H2: Message content will impact the user perception of digital nudges.
%\item H3: Prior activity level impacts the perceived effectiveness of digital nudging
%\end{itemize}

\section{Aims}
Primary Aim: Looking at message framing in digital nudging for physical activity

Secondary Aim: Looking at users experience with digital nudging when implemented as push notification integrated in a gym app 

Looking at how a certain implementation of digital nudging impacts attitudes towards regular physical activity by collecting qualitative data about users experience and perception 


\section{Objectives}
Objective: 
\begin{itemize}
\item To provide insight that can help answer the "how to nudge"-question 
\item To describe the user experience of this certain digital nudging context 
\item To gather insights about the effect of message framing in digital nudging 
\item To provide an evaluation of health informational nudges in the gym-context
\item 
\end{itemize}

\section{Covid-19}

\section{Delimitations}
The target population for this study are individuals ... 

The study was carried out in the Oslo area in Norway, 

\section{Context and Participants}
\textbf{    Target population and study sample}
%Aka: targeted users or population of interest.skal flyttes til intro? 
Individer som oppfyller to av tre faktorer for å oppnå atferden, fysisk aktivitet. Ability and motivation. 

Should I explain why they are the target population, or just describe the target population? 

Based on the prerequisites about presence of ability and motivation, the target population includes almost (except people with injuries that stops them from engaging in physical activity or those who pay for the membership without any kind of motivation) everyone who is a member of a training center in Norway, which is a rather large group of people. It would be demanding to reach out for all of them, if not impossible. Therefore, there was made some frames and limitations in order to form a sample of the population. The sample population for this study consists of people with membership at SATS and that has downloaded their membership-app. 
 
 The study sample is representative to its target population to some degree, as the participants varies in gender, age, motivation type, prior frequency of physical activity, attitudes, etc. (However, they do all belong to the same geographical area, as this was one of the frames for the sample).
 
\section{Research design and context}
\subsection{Context}
This study uses a qualitative research design (answering the whys and hows), where data are collected through interviews. 

The study is conducted in 2020, in affiliation with the Masters of Human-Computer Interaction programme in the Department of technology at Kristiania University College. This master thesis is conducted in collaboration with SATS, the biggest fitness center in Norway, to investigate the defined problem statement in a context with gym members. SATS have gym centres all over Scandinavia, which makes them a useful partner which offers an opportunity to include a wider target population across country borders. However, considering the scope of this master thesis, the study was limited to only including norwegian members due to physical attendance for interviews. 

Through this study I am conducting an experiment where technology mediated / digital nudges is tested in a real-life context with gym members of SATS. The nudges will be sent as a push notification through the app that SATS distributes to their members. The push notification will only include text and no other visual effect. The SATS app main functionality is membership identification (QR scan to enter gym centers), booking group sessions and log training. However, it can be categorized as a persuasive mobile application as they are implementing different elements in order to increase motivation for physical activity, and to make their members become regular trainers. Such elements is for instance: challenges, social comparison, feedback, and others. 

\subsection{Participants}

%\section{Stakeholders} Ha med eller ikke?



